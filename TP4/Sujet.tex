\documentclass[11pt,a4paper]{article}
\usepackage{graphicx}
\usepackage{longtable}
\usepackage{float}
\usepackage{wrapfig}
\usepackage{rotating}
%\usepackage{a4wide}
\usepackage[normalem]{ulem}
\usepackage{amsmath}
\usepackage{textcomp}
\usepackage{marvosym}
\usepackage{wasysym}
\usepackage{amssymb}
\usepackage{hyperref}
\usepackage{array}
%\usepackage[utf8]{inputenc}
%\usepackage[T1]{fontenc}
%\usepackage{polyglossia}
%\setmainlanguage[]{french}
\usepackage[top=30pt,bottom=30pt,left=48pt,right=46pt]{geometry}
\usepackage{listings}

\lstset{
    language=C++
  , frame=single
  , basicstyle=\scriptsize
  , keywordstyle=\color{blue}\bfseries
  , commentstyle=\color{red}
  , stringstyle=\color{magenta}\ttfamily
  , backgroundcolor=\color{black!5}
}

\usepackage[dvipsnames]{xcolor}
\usepackage{subfigure,tikz}
\usetikzlibrary{shadows,shadows.blur, trees,matrix,arrows,decorations}
\usetikzlibrary{decorations.pathmorphing, fadings, shapes, shapes.arrows, positioning, calc, shapes,fit}
\usepackage{fp}
\usepackage{pgfplots}
\tolerance=1000
\author{Xavier JUVIGNY}
\date{\today}
\title{Travaux dirigés n°4}

\newtheorem{definition}{Définition}
\newtheorem{theo}{Théorème}

\definecolor{verylightgray}{rgb}{0.95 0.95 1.0}

\definecolor{darkblue}{rgb}{0. 0. 0.4}

\begin{document}
\maketitle
\tableofcontents

\section{Présentation de TBB}

\subsection{Introduction}

La bibliothèque TBB (Threading Building Block library) est une bibliothèque C++ qui remplit deux rôles~:
\begin{enumerate}
    \item Remplit des fonctionnalités manquantes au support pour le parallélisme dans le standard C++, là où le langage n'a pas
          suffisamment évolué ou bien où les nouvelles caractéristiques du C++ ne sont pas encore bien supportées par les
          compilateurs;
    \item Propose une interface abstraite de haut niveau pour le parallélisme qui est au delà de ce que le standard C++ pourra
          proposer. 
\end{enumerate}

TBB contient un certain nombre de caractéristiques qu'on peut catégoriser en deux grands groupes :
\begin{itemize}
    \item Des interfaces pour exprimer des algorithmes parallèles;
    \item Des interfaces qui sont indépendantes du modèle d'exécution
\end{itemize}

Dans la première catégorie, on trouve :
\begin{itemize}
    \item Des algorithmes parallèles génériques tels que la parallélisation de boucle, les réductions, etc.
    \item Des algorithmes de parallélisation par tâche (on définit un graphe de tâches qui seront exécutés en parallèle si possible par TBB)
    \item Une version parallèle de la STL comme proposé par la norme C++17/C++20.
\end{itemize}

Dans la seconde catégorie, on trouve :
\begin{itemize}
    \item Des conteneurs supportant la concurrence en multithreading (Tables de hashage, queues, tableaux dynamiques)
    \item Des allocateurs parallélisables ou bien qui peuvent s'aligner avec les lignes de cache, etc.
    \item Des primitives de synchronisation, etc.
\end{itemize}

Nous nous intéresserons ici surtout à l'utilisation des algorithmes parallèles génériques, et en particulier à la parallélisation de boucle.

\subsection{Installation de TBB}

\begin{itemize}
\item Sous Windows (MSys 2) : \texttt{pacman -S mingw64\/mingw-w64-x86\_64-intel-tbb}
\item Sous Linux Ubuntu (WSL ou Linux natif) : \texttt{sudo apt install libtbb-dev}
\item Sous Mac-OS X (avec homebrew) : \texttt{brew install tbb}
\end{itemize}

Allez dans le répertoire \texttt{TBBExamples} et essayez de compiler les exemples
(en utilisant \texttt{make}).

\subsection{Parallélisme de boucle}

La parallélisation d'une boucle se fait à l'aide de la fonction \texttt{tbb::parallel\_for} qui se décline en plusieurs
versions :
\begin{itemize}
    \item Une version simple : \texttt{tbb::parallel\_for(start, end, step, lambda\_function)}. Par exemple :
    \begin{lstlisting}
        void print_hello( int i ) { std::cout << i << " : Hello world !" << std::endl; }
        
        int main()
        {   ...
            tbb::parallel_for(0, 100, 2, [](int i) { print_hello(i); });        
    \end{lstlisting}
    \item Une version utilisant des blocs en 1D, 2D ou 3D (voir plus loin).
    \item On peut également rajouter un partitioneur qui définira la stratégie de
          distribution des blocs et de leur répartion. Par défaut, le découpage essaye 
          d'être ``intelligent'' et la distribution des blocs est dynamique.
\end{itemize}

\subsubsection{Les blocs d'indices}

TBB propose des classes permettant de répartir les indices dans plusieurs tâches.
Il existe de base trois classes permettant de répartir en 1D, 2D ou 3D des données 
(la dimension correspond au nombre de boucles imbriquées). Il existe également
une classe générique permettant de gérer des dimensions supérieures.

Pour chaque indice géré par la classe, il faudra donner à sa construction
les indices de début et de fin de la boucle correspondante et,
en option, une taille de grain. L'utilisation de cette taille de grain 
diffère selon le partitionneur employé (on verra cela dans la section suivante).

La syntaxe pour les trois classes de base (templatées) est la suivante :
\begin{itemize}
    \item En une dimension : 
    \begin{lstlisting}
tbb::blocked_range<type indice>(start, end,grainSize=1)
    \end{lstlisting}
    Exemple d'utilisation (voir \texttt{compute\_pi.cpp} pour une utilisation concrète) :
    \begin{lstlisting}
tbb::blocked_range<int> r(0, 100, 2);
for (int i = r.begin(); i != r.end(); ++i )
{
    ...
}
    \end{lstlisting}
    \item En deux dimensions :
    \begin{lstlisting}
 tbb::blocked_range2d<type indice1, type indice2>(start i1, end i1, grainSize1, 
                                                  start i2, end i2, grainSize2);
    \end{lstlisting}
    On accède aux premiers indices à l'aide de la méthode \texttt{rows}
    et aux seconds indices à l'aide de la méthode \texttt{cols}.\\
    Exemple d'utilisation (voir \texttt{Mandelbrot\_tbb.cpp} pour un exemple concret) :
    \begin{lstlisting}
auto r = tbb::blocked_range2d<int,unsigned>(-10,10,1,0, 40, 2);
for (auto i = r.rows().begin(); i != r.rows().end(); ++i)
  for (auto j = r.cols().begin(); j != r.cols().end(); ++j)
  {
    ....
  }
    \end{lstlisting}
    \item En trois dimensions :
    \begin{lstlisting}
tbb::blocked_range3d<type indice1, type indice2, type indice3>(start i1, end i1, grainSize1,
                                                               start i2, end i2, grainSize2,
                                                               start i3, end i3, grainSize3);
    \end{lstlisting}
    On accède aux premiers indices à l'aide de la méthode \texttt{pages},
    aux seconds indices à l'aide de la méthode \texttt{rows} et 
    aux troisièmes indices à l'aide de la méthode \texttt{cols}.
    Exemple d'utilisation :
    \begin{lstlisting}
auto r = tbb::blocked_range3d<int,int,std::size_t>(-10,10,1,-10,10,1,0,40,2);
for (auto i = r.pages().begin(); i != r.pages().end(); ++i )
  for (auto j = r.rows().begin(); j != r.rows().end(); ++j )
    for (auto k = r.cols().begin(); k != r.cols().end(); ++k )
    {
      ...
    }
    \end{lstlisting}
\end{itemize}

\subsubsection{Les partitioneurs}

Un partitioneur permet de spécifier comment un algorithme
devra partitionner sa boucle. La table suivante résume les 
caractéristiques des différents partitioneurs (le grain correspond au 
nombre d'itération  que doit faire une boucle après avoir effectué
une partition de la boucle originale) :
\newcolumntype{M}[1]{>{\raggedright}m{#1}}

\begin{tabular}{|c|M{6cm}|M{4cm}|} \hline
\textbf{Partitioneur} & \textbf{Description} & \textbf{Avec des blocs d'indices} \tabularnewline \hline \hline
\texttt{simple\_partitioner} & Taille des blocs limitée par le grain & $\frac{g}{2} \leq \mbox{taille} \leq g$ \tabularnewline \hline 
\texttt{auto\_partitioner} & Calcul automatiquement la taille du grain & $\frac{g}{2} \leq \mbox{taille}$ \tabularnewline \hline
\texttt{static\_partitioner} & Détermine une taille de boucle fixe et une
distribution uniforme des itérations sans faire d'équilibrage des charges. La distribution 
uniforme est créée sauf si la taille du grain empêche de créer $P$ parties &
$\max(\frac{g}{3}, \frac{N}{P}) \leq \mbox{taille}$ où
\begin{itemize}
    \item $N$ est la taille du problème
    \item $P$ est le nombre de threads
\end{itemize} 
\tabularnewline \hline
\end{tabular}

Pour une utilisation des partitioneurs et des blocs d'indices dans une boucle
parallèle, allez regarder dans les exemples le code \texttt{Mandelbrot\_tbb.cpp}

\section{Exercices}

\subsection{Open MP et MPI}

Dans un premier temps, nous allons modifier l'exercice fait au deuxième TD sur Mandelbrot.

\begin{enumerate}
    \item Faites en sorte que désormais, une tâche MPI corresponde au calcul de $nt$ lignes
(et non d'une seule ligne).
    \item Parallélisez à l'aide d'OpenMP le calcul dans une tâche des $nt$ lignes à l'aide de 
    $nt$ threads (on rajoutera la clause \texttt{num\_threads}(nt) dans le \texttt{\#pragma omp})
    \item Regardez les différentes accélérations obtenues en fonction du nombre de processus MPI et 
    du paramètre $nt$ et regardez quel est le meilleur temps que vous obtenez.
    \item Comparez les temps de calcul obtenus pour deux processus (un maître et un esclave)
          avec les temps obtenus avec TBB pour $nt$ threads (en modifiant l'exemple : lignes à décommenter et modifier)
\end{enumerate}

\subsection{Parallélisation avec TBB}

Parallélisez à l'aide de TBB le code de raytracing fourni avec le TP.
Essayez divers partitioneurs et tailles de grain et montrez à l'aide des 
temps mesurés que par défaut, TBB fait bien de l'équilibrage de charge dynamique.
\end{document}
